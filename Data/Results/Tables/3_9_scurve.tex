\begin{center}
\begin{tabular}{lclc}
\toprule
\textbf{Dep. Variable:}             & total\_contribution & \textbf{  R-squared:         } &     0.381   \\
\textbf{Model:}                     &         OLS         & \textbf{  Adj. R-squared:    } &     0.381   \\
\textbf{Method:}                    &    Least Squares    & \textbf{  F-statistic:       } &     282.0   \\
\textbf{Date:}                      &   Mon, 20 Nov 2023  & \textbf{  Prob (F-statistic):} & 7.19e-165   \\
\textbf{Time:}                      &       19:29:40      & \textbf{  Log-Likelihood:    } &    3603.4   \\
\textbf{No. Observations:}          &          3710       & \textbf{  AIC:               } &    -7199.   \\
\textbf{Df Residuals:}              &          3706       & \textbf{  BIC:               } &    -7174.   \\
\textbf{Df Model:}                  &             3       & \textbf{                     } &             \\
\textbf{Covariance Type:}           &         HAC         & \textbf{                     } &             \\
\bottomrule
\end{tabular}
\begin{tabular}{lcccccc}
                                    & \textbf{coef} & \textbf{std err} & \textbf{z} & \textbf{P$> |$z$|$} & \textbf{[0.025} & \textbf{0.975]}  \\
\midrule
\textbf{const}                      &       2.5664  &        0.115     &    22.359  &         0.000        &        2.341    &        2.791     \\
\textbf{revenue\_scurve\_diff}      &      -2.3992  &        0.130     &   -18.428  &         0.000        &       -2.654    &       -2.144     \\
\textbf{costs\_scurve\_diff}        &      -2.1112  &        0.113     &   -18.724  &         0.000        &       -2.332    &       -1.890     \\
\textbf{contribution\_scurve\_diff} &       0.3829  &        0.023     &    16.696  &         0.000        &        0.338    &        0.428     \\
\bottomrule
\end{tabular}
\begin{tabular}{lclc}
\textbf{Omnibus:}       & 597.032 & \textbf{  Durbin-Watson:     } &    1.896  \\
\textbf{Prob(Omnibus):} &   0.000 & \textbf{  Jarque-Bera (JB):  } & 7845.097  \\
\textbf{Skew:}          &   0.333 & \textbf{  Prob(JB):          } &     0.00  \\
\textbf{Kurtosis:}      &  10.093 & \textbf{  Cond. No.          } &     88.2  \\
\bottomrule
\end{tabular}
%\caption{OLS Regression Results}
\end{center}

Notes: \newline
 [1] Standard Errors are heteroscedasticity and autocorrelation robust (HAC) using 3 lags and without small sample correction